% resume template adapted from Aras Güngöre

\documentclass[letterpaper,11pt]{article}

\usepackage{latexsym}
\usepackage[empty]{fullpage}
\usepackage{titlesec}
\usepackage{marvosym}
\usepackage[dvipsnames]{xcolor}
\usepackage{verbatim}
\usepackage{enumitem}

\usepackage{hyperref}
\hypersetup{
    colorlinks = true,
    allcolors = Brown,
}

\usepackage{fancyhdr}
\usepackage[english]{babel}
\usepackage{tabularx}
\usepackage{hyphenat}
\usepackage{fontawesome}
\usepackage{ragged2e}
% fix linespacing
\usepackage{microtype}
\renewcommand{\baselinestretch}{0.95}
\usepackage[super]{nth}

\input{glyphtounicode}

\newcommand*{\doi}[1]{\href{http://dx.doi.org/#1}{doi:#1}}
\newcommand*{\arxiv}[1]{\href{http://arxiv.org/abs/#1}{arXiv:#1}}
\newcommand*{\biorxiv}[1]{\href{http://biorxiv.org/content/#1}{bioRxiv:#1}}
\newcommand*{\pubmed}[1]{\href{http://www.ncbi.nlm.nih.gov/pubmed/#1}{PMID:#1}}


%---------- FONT OPTIONS ----------
% sans-serif
% \usepackage[sfdefault]{FiraSans}
% \usepackage[sfdefault]{roboto}
% \usepackage[sfdefault]{noto-sans}
% \usepackage[default]{sourcesanspro}

% serif
% \usepackage{CormorantGaramond}
\usepackage[T1]{fontenc}
\usepackage[bitstream-charter]{mathdesign}


\pagestyle{fancy}
\fancyhf{} % clear all header and footer fields
\fancyfoot{}
\renewcommand{\headrulewidth}{0pt}
\renewcommand{\footrulewidth}{0pt}

% Adjust margins
\addtolength{\oddsidemargin}{-0.5in}
\addtolength{\evensidemargin}{-0.5in}
\addtolength{\textwidth}{1in}
\addtolength{\topmargin}{-.5in}
\addtolength{\textheight}{1.0in}

\urlstyle{same}

\raggedbottom
\raggedright
\setlength{\tabcolsep}{0in}

% Sections formatting
\titleformat{\section}{
  \vspace{-4pt}\scshape\raggedright\large
}{}{0em}{}[\color{black}\titlerule \vspace{-5pt}]

% Ensure that generate pdf is machine readable/ATS parsable
\pdfgentounicode=1

%-------------------------
% Custom commands

\newcommand{\resumeItem}[1]{
  \item\small{
    {#1 \vspace{-2pt}}
  }
}

\newcommand{\resumeSubSubheading}[2]{
    \vspace{-2pt}\item
    \begin{tabular*}{0.97\textwidth}{l@{\extracolsep{\fill}}r}
      \textit{\small#1} & \textit{\small #2} \\
    \end{tabular*}\vspace{-7pt}
}


\newcommand{\resumeSubheading}[5]{
  \vspace{-2pt}\item
    \begin{tabular*}{0.97\textwidth}[t]{l@{\extracolsep{\fill}}r}
      \textbf{#1} & #2 \\
      \textit{\small#3} & \textit{\small #4} \\
    \end{tabular*}
        \ifx\relax#5\relax 
          \vspace{-7pt}
        \else
          \linebreak\textit{\small#5}
          \vspace{-3pt}
        \fi
}

\newcommand{\simpleHeading}[3]{
  \vspace{-2pt}\item
    \begin{tabular*}{0.97\textwidth}[t]{l@{\extracolsep{\fill}}r}
      \textbf{\small#1} & \small#2 \\
    \end{tabular*}
    \ifx\relax#3\relax 
    \else
      \linebreak\textit{\small#3}
    \fi
    \vspace{-5pt}
}

\newcommand{\resumeSubHeadingListStart}{\begin{itemize}[leftmargin=0.10in, rightmargin=0.10in, label={}]}
\newcommand{\resumeSubHeadingListEnd}{\end{itemize}}
\newcommand{\resumeItemListStart}{\begin{itemize}}
\newcommand{\resumeItemListEnd}{\end{itemize}\vspace{-5pt}}

\begin{document}
\justifying

\begin{center}
  \textbf{\Huge \scshape Jason Ken Adhinarta} \\ \vspace{3pt}
    \small
    \href{https://jasonkena.github.io}{jasonkena.github.io}
   \hspace{0.05cm}$\cdot$\hspace{0.05cm}
    \href{mailto:jason.adhinarta@bc.edu}{ jason.adhinarta@bc.edu }
   \hspace{0.05cm}$\cdot$\hspace{0.05cm}
    Chestnut Hill, MA
\end{center}


\section{Research Experience}
  \vspace{3pt}
  \resumeSubHeadingListStart
    
    \resumeSubheading
    {Boston College Computer Vision Lab}{Chestnut Hill, MA}{Research Assistant (advised by Prof.\,Donglai Wei)}{Sep 2021 \textbf{--} Present}{}
      \resumeItemListStart
      
        \resumeItem{\textit{Ongoing}: Crafted object-centric constraints for \textbf{terabyte-scale superpixel agglomeration} methods for \textbf{PyTorch Connectomics} using the SegCLR (\textbf{segmentation-guided contrastive learning}) pretrained model, in an attempt to build an open source competitor to Google's Flood Filling Networks. In collaboration with Zengyu Yang}
      
        \resumeItem{\hypertarget{hydra_bio_back}{\hypertarget{hydra_back}{\textit{Ongoing}: Clustered vesicles in \textit{Hydra vulgaris} samples using PyroVED \textbf{translation/rotation equivariant autoencoders}, characterizing neurons using spatial distribution of vesicles. Part of larger work to build a toolbox for \textbf{vesicle analysis} for electron microscopy. In collaboration with Shulin Zhang from Prof.\,Rafael Yuste's lab at Columbia University {[A]} {[B]}}}}
      
        \resumeItem{\hypertarget{wormnd_back}{\textit{Ongoing}: Adapted \textbf{cell tracking models} (3DeeCellTracker and Ultrack) to extract \textbf{whole-brain neural dynamics} from \textbf{calcium imaging} of 118 NeuroPAL-strain roundworms across 5 labs. Work done in collaboration with Daniel Sprague from the UCSF Foundations of Cognition Lab, Prof.\,Erdem Varol at the NYU Neuroinformatics Lab, and Prof.\,Eviatar Yemini at UMass Chan Medical School {[C]}}}
      
        \resumeItem{\hypertarget{freseg_back}{Benchmarked PointNet++, RandLA-Net, and PointTransformer \textbf{point-cloud segmentation architectures} on our dataset of 4,476 spines sourced from electron microscopy of 70 dendrites, demonstrating \textbf{zero-shot cross-species generalization} of our proposed Frenet-Serret \textbf{equivariant geometric transform} on the mouse visual cortex ($94.1\%$ Dice) and human frontal lobe ($81.8\%$ Dice). Also wrote extensive pipelines for morphology processing and mesh visualization. Part of a project to improve \textbf{segmentation on tree-like structures}; locating synapses in neurons and aneurysms in blood vessels. Collaborated with Shixuan Gu at the Harvard Visual Computing Group {[D]}}}
      
        \resumeItem{\hypertarget{trisam_back}{Analyzed \textbf{blood vessel morphologies} in mouse, human, and macaque cortical samples using Kimimaro \textbf{centerline extraction}. Evaluated the performance of the U-Net3D baseline model on the largest electron microscopy blood vessel dataset to date. Part of a project to develop a \textbf{zero-shot 3D segmentation method} using \textbf{Segment Anything Model}. In collaboration with Prof.\,Jia Wan from Harbin Institute of Technology, China, the Harvard Visual Computing Group, and the Harvard Lichtman Lab {[E]}}}
      
        \resumeItem{\hypertarget{xiaomeng_back}{Finetuned Cellpose, a \textbf{{foundation model for cell segmentation}}, to automatically detect \textasciitilde{}5 million neuronal vesicles in 20 volumes. Work done with Dr.\,Xiaomeng Han from Prof.\,Jeff Lichtman’s lab at Harvard to develop a novel microscopy technique incorporating fluorescent markers to facilitate \textbf{identification of cell types and functions}. {[F]}}}
      
        \resumeItem{Prototyped methods to correct deformed \textbf{spinal vertebrae mesh predictions} using Probreg \textbf{point-cloud registration} for Dr.\,Zongwei Zhou at JHU}
      
        \resumeItem{Maintained \textbf{Docker evaluation containers} for the ISBI 2013 SNEMI3D neuron segmentation challenge, MICCAI 2021 AxonEM axon segmentation challenge, and the ISBI 2023 RNR-EXM expansion microscopy image registration challenge on the \textbf{Grand Challenge} platform, to simplify benchmark evaluation for 320+ participants}
      
        \resumeItem{Developed the ChunkPipeline package to perform distributed volumetric computations on the \textbf{Boston College Linux Cluster}, contributed bugfixes to the Princeton Seung Lab's suite of tools for connectomics, onboarded research interns onto the PyTorch Connectomics ecosystem}
      
      \resumeItemListEnd
      \filbreak
    
    \resumeSubheading
    {EPFL CVLab}{Lausanne, Switzerland}{Research Intern (advised by Dr.\,Jiancheng Yang and Prof.\,Pascal Fua)}{May 2023 \textbf{--} Aug 2023}{}
      \resumeItemListStart
      
        \resumeItem{\hypertarget{imheart_back}{Orchestrated pipelines integrating the STU-Net \textbf{medical image segmentation foundation model} with our \textbf{geometrically-constrained neural implicit fields} to generate anatomically accurate heart structures from our aggregated dataset of 140 \textbf{MRI scans}. Contributed to the Heart Augmented Reality Training System, in collaboration with Swiss medical imaging company ADIS and the Cardiology Division of Lausanne University Hospital {[G]}}}
      
        \resumeItem{\hypertarget{ribseg_back}{Evaluated \textbf{voxel/point-cloud segmentation} models (PointNet/PointNet++, DGCNN, PointCNN, nnU-Net) and \textbf{centerline extraction} techniques (Kimimaro, L1-Medial Skeletonization) on our proposed dataset of 660 \textbf{CT scans} and 15,466 individually annotated ribs. Implemented a Docker compatability wrapper for the L1-Medial Skeletonization codebase from 2016. Ongoing work to incorporate the \textbf{mesh representations of ribcages} into the \textbf{MedShapeNet2.0} dataset. In collaboration with radiologists at Huadong Hospital, China and Shixuan Gu at Harvard Visual Computing Group {[H]}}}
      
      \resumeItemListEnd
      \filbreak
    
    \resumeSubheading
    {Emmerich Research Center}{Jakarta, Indonesia}{Research Intern (advised by Dr.\,Eden Steven)}{Aug 2018 \textbf{--} Aug 2021}{}
      \resumeItemListStart
      
        \resumeItem{\hypertarget{phosphor_back}{Studied the \textbf{temperature-dependent excitation} curves of SrAl\textsubscript{2}O\textsubscript{4}:Eu\textsuperscript{2+}, Dy\textsuperscript{3+} glow-in-the-dark crystals. Built vacuum-sealed optical probes operating at cryogenic temperatures. In collaboration with Dr.\,Muhandis Shiddiq from Indonesia's National Research and Innovation Agency and Prof.\,Henri Uranus from Universitas Pelita Harapan {[I]}}}
      
        \resumeItem{\hypertarget{mycelium_back}{Developed an \textbf{OpenCV-based contamination detection system} featuring perspective normalization, shadow removal, and blob detection, to control a CNC-sprayer for disinfection of fungal cultures. Delivered a proof-of-concept robot for MycoWorks, a California-based startup producing \textbf{plant-based synthetic leather} {[J]}}}
      
        \resumeItem{Trained the YOLACT \textbf{instance segmentation} model to track Black Soldier Fly larvae. Wired an Arduino-controlled linear-slider system for recording larvae behavior in arrays of petri dishes. Set up CVAT data annotation pipelines. An effort to \textbf{optimize larvae feed and rearing conditions for waste-processing} in collaboration with Hermetia Bio Sciences}
      
        \resumeItem{Automated palm oil \textbf{fruit quality assurance} using OpenCV and XGBoost tree models. Deployed training pipelines on Google Cloud Platform. In collaboration with manufacturing company PT.\,Sawit Asahan Tetap Utuh}
      
        \resumeItem{Finetuned the YOLACT segmentation model on the TACO waste dataset, integrated with a 3-DOF robotic arm with a gradient-descent-based \textbf{inverse kinematics} solver. Presented at Indonesia Science Expo 2019}
      
        \resumeItem{Computationally modeled angle-dependent Ohmic resistance of stacked hexagonal lattices analogous to twisted bilayer graphene using \textbf{CUDA-accelerated sparse linear solvers}, attempting to understand magic angle superconductivity}
      
        \resumeItem{Co-designed a 23-week long \textbf{Arduino-based electronics programming curriculum} for Sekolah Pelangi Kasih's afterschool program. Co-instructed the electronics portion of Sekolah Pelita Harapan's 2019 Summer Science Academy}
      
      \resumeItemListEnd
      \filbreak
    
  \resumeSubHeadingListEnd




\end{document}
