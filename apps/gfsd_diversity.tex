% resume template adapted from Aras Güngöre

\documentclass[letterpaper,11pt]{article}

\usepackage{latexsym}
\usepackage[empty]{fullpage}
\usepackage{titlesec}
\usepackage{marvosym}
\usepackage[dvipsnames]{xcolor}
\usepackage{verbatim}
\usepackage{enumitem}

\usepackage{hyperref}
\hypersetup{
    colorlinks = true,
    allcolors = Brown,
}

\usepackage{fancyhdr}
\usepackage[english]{babel}
\usepackage{tabularx}
\usepackage{hyphenat}
\usepackage{fontawesome}
\usepackage{ragged2e}
\usepackage[super]{nth}
\usepackage{expl3}



\ExplSyntaxOn
% Define the custom command
\NewDocumentCommand{\customcommand}{mm}{
    \int_zero:N \l_tmpa_int % Reset the counter
    % Parse the input lists
    \seq_set_split:Nnn \l_tmpa_seq { , } { #1 } % First list
    \seq_set_split:Nnn \l_tmpb_seq { , } { #2 } % Second list

    % Ensure both lists have the same length
    \int_compare:nNnTF {\seq_count:N \l_tmpa_seq} = {\seq_count:N \l_tmpb_seq}
        {
            % If lengths match, generate hyperlinks
          [\seq_mapthread_function:NNN \l_tmpa_seq \l_tmpb_seq \__customcommand_generate_hyperlink:nn]
        }
        {
            % If lengths don't match, throw an error
            \textbf{Error: Lists must have the same length!}
        }
}

\cs_new_protected:Nn \__customcommand_generate_hyperlink:nn {
    \int_incr:N \l_tmpa_int % Increment the counter
    \hyperlink{#1}{#2}%
    % Add a comma and space unless this is the last item
    \int_compare:nNnT {\l_tmpa_int} < {\seq_count:N \l_tmpa_seq}
        {,~}
}

\ExplSyntaxOff


\input{glyphtounicode}

\newcommand*{\doi}[1]{\href{http://dx.doi.org/#1}{doi:#1}}
\newcommand*{\arxiv}[1]{\href{http://arxiv.org/abs/#1}{arXiv:#1}}
\newcommand*{\biorxiv}[1]{\href{http://biorxiv.org/content/#1}{bioRxiv:#1}}
\newcommand*{\pubmed}[1]{\href{http://www.ncbi.nlm.nih.gov/pubmed/#1}{PMID:#1}}


%---------- FONT OPTIONS ----------
% sans-serif
% \usepackage[sfdefault]{FiraSans}
% \usepackage[sfdefault]{roboto}
% \usepackage[sfdefault]{noto-sans}
% \usepackage[default]{sourcesanspro}

% serif
% \usepackage{CormorantGaramond}
\usepackage[T1]{fontenc}
\usepackage[bitstream-charter]{mathdesign}


\pagestyle{fancy}
\fancyhf{} % clear all header and footer fields
\fancyfoot{}
\renewcommand{\headrulewidth}{0pt}
\renewcommand{\footrulewidth}{0pt}

% Adjust margins
% \addtolength{\oddsidemargin}{-0.5in}
% \addtolength{\evensidemargin}{-0.5in}
% \addtolength{\textwidth}{1in}
% \addtolength{\topmargin}{-.5in}
% \addtolength{\textheight}{1.0in}
\addtolength{\oddsidemargin}{-0.25in}
\addtolength{\evensidemargin}{-0.25in}
\addtolength{\textwidth}{0.5in}
\addtolength{\topmargin}{-.5in}
\addtolength{\textheight}{1.0in}

\urlstyle{same}

\raggedbottom
\raggedright
\setlength{\tabcolsep}{0in}

% Sections formatting
\titleformat{\section}{
  \vspace{-4pt}\scshape\raggedright\large
}{}{0em}{}[\color{black}\titlerule \vspace{-5pt}]


\newcommand{\dualsectionold}[2]{%
  \noindent
    \parbox[b]{0.5\textwidth}{\raggedright\scshape\large #1}%
    \hfill
    \parbox[b]{0.5\textwidth}{\raggedleft\scshape\large #2}%
}
\newcommand{\dualsectionhline}{%
  \vspace{2pt}\color{black}\titlerule\vspace{-5pt}
}
  
\newcommand{\dualsection}[2]{%
  \dualsectionold{#1}{#2}%
  \vspace{-5pt}\dualsectionhline
}

% Ensure that generate pdf is machine readable/ATS parsable
\pdfgentounicode=1

%-------------------------
% Custom commands

\newcommand{\resumeItem}[1]{
  \item\small{
    {#1 \vspace{-2pt}}
  }
}

\newcommand{\resumeSubSubheading}[2]{
    \vspace{-2pt}\item
    \begin{tabular*}{0.97\textwidth}{l@{\extracolsep{\fill}}r}
      \textit{\small#1} & \textit{\small #2} \\
    \end{tabular*}\vspace{-7pt}
}


\newcommand{\resumeSubheading}[5]{
  \vspace{-2pt}\item
    \begin{tabular*}{0.97\textwidth}[t]{l@{\extracolsep{\fill}}r}
      \textbf{#1} & #2 \\
      \textit{\small#3} & \textit{\small #4} \\
    \end{tabular*}
        \ifx\relax#5\relax 
          \vspace{-7pt}
        \else
          \linebreak\textit{\small#5}
          \vspace{-3pt}
        \fi
}

\newcommand{\simpleHeading}[3]{
  \vspace{-2pt}\item
    \begin{tabular*}{0.97\textwidth}[t]{l@{\extracolsep{\fill}}r}
      \textbf{\small#1} & \small#2 \\
    \end{tabular*}
    \ifx\relax#3\relax 
    \else
      \linebreak\textit{\small#3}
    \fi
    \vspace{-5pt}
}

\newcommand{\resumeSubHeadingListStart}{\begin{itemize}[leftmargin=0.10in, rightmargin=0.10in, label={}]}
\newcommand{\resumeSubHeadingListEnd}{\end{itemize}}
\newcommand{\resumeItemListStart}{\begin{itemize}}
\newcommand{\resumeItemListEnd}{\end{itemize}\vspace{-5pt}}

\begin{document}
\justifying

\begin{center}
  \textbf{\Huge \scshape Jason Ken Adhinarta} \\ \vspace{3pt}
    \small
    \href{https://jasonkena.github.io}{jasonkena.github.io}
   \hspace{0.05cm}$\cdot$\hspace{0.05cm}
    \href{mailto:jason.adhinarta@bc.edu}{ jason.adhinarta@bc.edu }
   \hspace{0.05cm}$\cdot$\hspace{0.05cm}
    Chestnut Hill, MA
\end{center}


\dualsection{Diversity Statement}{Graduate Fellowships for STEM Diversity}
\vspace{2pt}\color{black}\titlerule%\vspace{-5pt}

I am indebted to the researchers in Indonesia who would at times drive through gridlock traffic to alleviate my struggles with shorthand matrix multiplication notation, as well as the PhD student in Switzerland who lent me his rice cooker when I short-circuited mine. I strive to embody their enthusiasm and patience in my own mentoring.

Since Fall 2023, I have been \textbf{remotely advising students} at Sekolah Pelita Harapan in Tangerang, Indonesia through the \textbf{Applied Science Academy}. This year, I have the pleasure of working with a sophomore interested in autonomous vehicles. Months ago, we brainstormed the idea of building a remote-controlled car with a mounted camera to create spatial navigation maps using neural radiance fields. Last week, he set up a Flask server on a Raspberry Pi to control the car’s motors, and he is now working on feeding the camera footage into Nerfstudio. Throughout the process, I designed lesson plans catered to his programming level, assisted with debugging, and guided him through literature search---working hard to nurture his curiosity rather than overshadow it. It was through the same program that I found my passion for computer vision. I am humbled to now support his growth in robotics, hoping to help broaden his educational and career opportunities. 

My interests in mentorship have also brought me to lead the \textbf{Machine Intelligence Group at Boston College}---a community of students who read arXiv preprints for fun. My favorite moments have been the goodbyes at the stairwell which devolved into hour-long debates, one of which was about whether the Planning Domain Definition Language could “completely solve” rock climbing, akin to what AlphaZero did for chess. Moments like these have demonstrated to me that fostering genuine curiosity extends beyond studying theoretical concepts; ultimately, it is about engaging with real-world problems that matter on a personal level. \textbf{I believe that bridging the gap between students and impactful engineering involves real exploration beyond sanitized problem sets}.

My experiences dealing with \textbf{financial constraints as a first-generation American} from Indonesia have instilled in me a deep commitment to making education more accessible. I am grateful for the full-tuition scholarship and need-based aid that made my undergraduate studies possible, and I have taken on research or teaching roles every semester to cover the remaining costs. \textbf{I aspire to become a professor} to follow my passions for both research and teaching, aiming to create opportunities for those who might face similar barriers.

Pursuing a PhD in Computer Science would be a step in that direction. I look forward to taking advantage of the service opportunities available during my doctoral studies. Whether by mentoring high school students through programs like \textbf{Project Teach in Boston} or being a \textbf{TA for introductory computer vision courses}, I hope to inspire.

I have come a long way from rigging cameras onto conveyor belts at the Emmerich Research Center, where we monitored larval growth one day and helped students build toy cars the next. And yet, today I find myself in a familiar role: computing neural activity traces for fluorescent roundworms by day and guiding a high schooler as he codes his room-mapping car by night---once again juggling responsibilities as a researcher and a mentor.



\end{document}
