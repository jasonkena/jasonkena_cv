% resume template adapted from Aras Güngöre

\documentclass[letterpaper,11pt]{article}

\usepackage{latexsym}
\usepackage[empty]{fullpage}
\usepackage{titlesec}
\usepackage{marvosym}
\usepackage[dvipsnames]{xcolor}
\usepackage{verbatim}
\usepackage{enumitem}

\usepackage{hyperref}
\hypersetup{
    colorlinks = true,
    allcolors = Brown,
}

\usepackage{fancyhdr}
\usepackage[english]{babel}
\usepackage{tabularx}
\usepackage{hyphenat}
\usepackage{fontawesome}
\usepackage{ragged2e}
\usepackage[super]{nth}
\usepackage{expl3}



\ExplSyntaxOn
% Define the custom command
\NewDocumentCommand{\customcommand}{mm}{
    \int_zero:N \l_tmpa_int % Reset the counter
    % Parse the input lists
    \seq_set_split:Nnn \l_tmpa_seq { , } { #1 } % First list
    \seq_set_split:Nnn \l_tmpb_seq { , } { #2 } % Second list

    % Ensure both lists have the same length
    \int_compare:nNnTF {\seq_count:N \l_tmpa_seq} = {\seq_count:N \l_tmpb_seq}
        {
            % If lengths match, generate hyperlinks
          [\seq_mapthread_function:NNN \l_tmpa_seq \l_tmpb_seq \__customcommand_generate_hyperlink:nn]
        }
        {
            % If lengths don't match, throw an error
            \textbf{Error: Lists must have the same length!}
        }
}

\cs_new_protected:Nn \__customcommand_generate_hyperlink:nn {
    \int_incr:N \l_tmpa_int % Increment the counter
    \hyperlink{#1}{#2}%
    % Add a comma and space unless this is the last item
    \int_compare:nNnT {\l_tmpa_int} < {\seq_count:N \l_tmpa_seq}
        {,~}
}

\ExplSyntaxOff


\input{glyphtounicode}

\newcommand*{\doi}[1]{\href{http://dx.doi.org/#1}{doi:#1}}
\newcommand*{\arxiv}[1]{\href{http://arxiv.org/abs/#1}{arXiv:#1}}
\newcommand*{\biorxiv}[1]{\href{http://biorxiv.org/content/#1}{bioRxiv:#1}}
\newcommand*{\pubmed}[1]{\href{http://www.ncbi.nlm.nih.gov/pubmed/#1}{PMID:#1}}


%---------- FONT OPTIONS ----------
% sans-serif
% \usepackage[sfdefault]{FiraSans}
% \usepackage[sfdefault]{roboto}
% \usepackage[sfdefault]{noto-sans}
% \usepackage[default]{sourcesanspro}

% serif
% \usepackage{CormorantGaramond}
\usepackage[T1]{fontenc}
\usepackage[bitstream-charter]{mathdesign}


\pagestyle{fancy}
\fancyhf{} % clear all header and footer fields
\fancyfoot{}
\renewcommand{\headrulewidth}{0pt}
\renewcommand{\footrulewidth}{0pt}

% Adjust margins
% \addtolength{\oddsidemargin}{-0.5in}
% \addtolength{\evensidemargin}{-0.5in}
% \addtolength{\textwidth}{1in}
% \addtolength{\topmargin}{-.5in}
% \addtolength{\textheight}{1.0in}
\addtolength{\oddsidemargin}{-0.25in}
\addtolength{\evensidemargin}{-0.25in}
\addtolength{\textwidth}{0.5in}
\addtolength{\topmargin}{-.5in}
\addtolength{\textheight}{1.0in}

\urlstyle{same}

\raggedbottom
\raggedright
\setlength{\tabcolsep}{0in}

% Sections formatting
\titleformat{\section}{
  \vspace{-4pt}\scshape\raggedright\large
}{}{0em}{}[\color{black}\titlerule \vspace{-5pt}]


\newcommand{\dualsectionold}[2]{%
  \noindent
    \parbox[b]{0.5\textwidth}{\raggedright\scshape\large #1}%
    \hfill
    \parbox[b]{0.5\textwidth}{\raggedleft\scshape\large #2}%
}
\newcommand{\dualsectionhline}{%
  \vspace{2pt}\color{black}\titlerule\vspace{-5pt}
}
  
\newcommand{\dualsection}[2]{%
  \dualsectionold{#1}{#2}%
  \vspace{-5pt}\dualsectionhline
}

% Ensure that generate pdf is machine readable/ATS parsable
\pdfgentounicode=1

%-------------------------
% Custom commands

\newcommand{\resumeItem}[1]{
  \item\small{
    {#1 \vspace{-2pt}}
  }
}

\newcommand{\resumeSubSubheading}[2]{
    \vspace{-2pt}\item
    \begin{tabular*}{0.97\textwidth}{l@{\extracolsep{\fill}}r}
      \textit{\small#1} & \textit{\small #2} \\
    \end{tabular*}\vspace{-7pt}
}


\newcommand{\resumeSubheading}[5]{
  \vspace{-2pt}\item
    \begin{tabular*}{0.97\textwidth}[t]{l@{\extracolsep{\fill}}r}
      \textbf{#1} & #2 \\
      \textit{\small#3} & \textit{\small #4} \\
    \end{tabular*}
        \ifx\relax#5\relax 
          \vspace{-7pt}
        \else
          \linebreak\textit{\small#5}
          \vspace{-3pt}
        \fi
}

\newcommand{\simpleHeading}[3]{
  \vspace{-2pt}\item
    \begin{tabular*}{0.97\textwidth}[t]{l@{\extracolsep{\fill}}r}
      \textbf{\small#1} & \small#2 \\
    \end{tabular*}
    \ifx\relax#3\relax 
    \else
      \linebreak\textit{\small#3}
    \fi
    \vspace{-5pt}
}

\newcommand{\resumeSubHeadingListStart}{\begin{itemize}[leftmargin=0.10in, rightmargin=0.10in, label={}]}
\newcommand{\resumeSubHeadingListEnd}{\end{itemize}}
\newcommand{\resumeItemListStart}{\begin{itemize}}
\newcommand{\resumeItemListEnd}{\end{itemize}\vspace{-5pt}}

\begin{document}
\justifying

\begin{center}
  \textbf{\Huge \scshape Joint Program Statement of Purpose} \\ \vspace{3pt}
    \small
    \href{https://jasonkena.github.io}{jasonkena.github.io}
   \hspace{0.05cm}$\cdot$\hspace{0.05cm}
    \href{mailto:jason.adhinarta@bc.edu}{ jason.adhinarta@bc.edu }
   \hspace{0.05cm}$\cdot$\hspace{0.05cm}
    Chestnut Hill, MA
\end{center}


\dualsection{Jason Ken Adhinarta}{Maryland-Max-Planck PhD Program}
\vspace{2pt}\color{black}\titlerule%\vspace{-5pt}


The Emmerich Research Center was located in the same building as a tutoring center in Jakarta, Indonesia. The lab members there juggled responsibilities as both scientists and teachers; one day rigging conveyor-belt-mounted webcams to monitor larval growth and another helping kids design heat-seeking autonomous cars to extinguish candles. Who wouldn't be intrigued?

In high school, as an intern for the lab, I learned how to code by designing temperature control systems for vacuum chambers, and was introduced to computer vision while attempting to calculate reaction rates from video footage of our weighing scales. Working alongside food scientists, electrical engineers, and physicists taught me early on that \textbf{the best kind of research is interdisciplinary}---requiring learning and teaching across fields to tackle complex problems.

I believe that the \textbf{Maryland-Max-Planck PhD program} offers a unique opportunity to explore how \textbf{computer vision} methods can be used to accelerate progress in \textbf{computational biology}, while promoting cross-cultural exchange.

As an Indonesian-American, I spent over a decade living in the Greater Jakarta area prior to my undergraduate studies. During my last three years there, I worked under the mentorship of Dr.\ Eden Steven on projects tailored to the unique challenges of Southeast Asian industries. These included automating palm oil fruit quality assurance pipelines for the local manufacturing company PT. Sawit Asahan Tetap Utuh and collaborating with Prof.\ Andreas Vilcinskas from Justus Liebig University Giessen, Germany, to track black soldier fly larvae behavior and optimize feed formulas.

As an undergraduate at Boston College, I worked closely with researchers at the Harvard Visual Computing Group on microscopy challenges. More recently, I collaborated with Dr.\ Xiaomeng Han from the Harvard Lichtman Lab, who pioneered a novel technique integrating fluorescent markers into electron microscopy to identify cell types and functions. This work, published in \textit{Nature Communications}, highlighted the importance of bringing together expertise from different organizations to drive breakthroughs in connectomics.

As a PhD student in the Maryland-Max-Planck program, I would love the opportunity to collaborate with \textbf{Prof.\ Heng Huang} at the University of Maryland and \textbf{Prof.\ Ivo Sbalzarini} at the Max Planck Institute of Molecular Cell Biology and Genetics. In particular, I hope to learn from Prof.\ Huang’s expertise in developing robust deep learning systems across histology, neuroimaging, and transcriptomics, as well as from Prof.\ Sbalzarini’s work on modeling and simulating tissue growth.

I am confident that the diverse perspectives brought by the Maryland-Max-Planck program will spark innovative approaches in both computational biology and computer vision.


\end{document}
